\documentclass[12pt,a4paper]{article}
\usepackage[utf8]{inputenc}% ermöglich die direkte Eingabe der Umlaute 
\usepackage[T1]{fontenc} % das Trennen der Umlaute
\usepackage[ngerman]{babel} % hiermit werden deutsche Bezeichnungen genutzt 
\usepackage{csquotes} 	% Gänsefüßchen nach gesetzter Sprache
\usepackage{graphicx}	% Bilder
\usepackage{subcaption}	% Für die Subfigures
\usepackage{geometry}	% Seitengröße
\usepackage[backend=biber,style=numeric,]{biblatex}	% Fürs Literatursverzeichis
\usepackage{glossaries}\makeglossaries
\geometry{
  left=3cm,
  right=3cm,
  top=2cm,
  bottom=4cm,
  bindingoffset=5mm
}
\loadglsentries{glossary.bib}
\addbibresource{literatur.bib}

%------------------------
% Deckblatt Titel und co.
%------------------------

\title{Praxisbericht\thanks{Damke am m1 Mamer uwu}}
\author{R\"uckblick auf die Semester 1 und 2\\
	Studiengang Informatik\\
	Studienrichtung Informatik\\
	DHBW Karlsruhe\\
	von Jakob Jonathan Heitzmann
}
\date{\today}

\begin{document}

%-------
% Bilder
%-------

\begin{figure}[t!]
	\begin{subfigure}{0.4\linewidth}
		\includegraphics[scale=0.25]{DHBWKarlsruhe.jpg}
	\end{subfigure}
	\hspace*{\fill}
	\begin{subfigure}{0.4\linewidth}
    	\includegraphics[scale=0.25]{YellowMap.png}
	\end{subfigure} 
\end{figure}

\maketitle
\vspace*{\fill}

%----------------
% Deckblatt unten
%----------------

\begin{center}
	\begin{tabular}{p{8cm}r}
		Matrikelnummer & 9119328\\
		Kurs & TINF19B4\\
		Ausbildungsfirma & YellowMap AG, Karlsruhe\\
		Betreuer & Markus Lind
	\end{tabular}
\end{center}
\newpage

%--------------------------
% Eidesstattliche Erklärung
%--------------------------

\section*{Eidesstattliche Erkl\"arung}

Gem\"a\ss\ \S\ 5 (3) der ''Studien- und Pr\"ufungsordnung DHBW Technik'' vom 29. September 2017 erkl\"are ich, Jakob Jonathan Heitzmann, dass ich die vorliegende Arbeit selbstst\"andig verfasst und keine anderen als die angegebenen Quellen und Hilfsmittel verwendet habe, sowie dass die physisch-vorliegende Version der Digitalen identisch ist.\\
\hspace*{0.5cm}
Karlsruhe, den \today

\vspace*{2cm}

\begin{tabular}{@{}l@{}}\hline
\rule{0pt}{2ex}
Jakob Jonathan Heitzmann
\end{tabular}
\newpage

%---------------
%Zusammenfassung
%---------------

\begin{abstract}
In dieser Arbeit geht es um meine Tätigkeit bei der YellowMap AG und was ich in den ersten zwei Semestern gelernt und getan habe. Diese Arbeit beschränkt sich aber in erster Linie auf die Bemühungen, die zur Umwandlung des Serversystems auf ein verteiltes System galten.
\end{abstract}
\newpage

\tableofcontents
\newpage

%-----------------------------------------------------------------------------------------------
%Hier beginnt das eigentliche Inhalt des Dokuments. (Davor nur Deckblatt bis Inhaltsverzeichniss)
%-----------------------------------------------------------------------------------------------

\section{Einf\"uhrung}
Ziel der Arbeit ist es vorzuzeigen was ich in den ersten beiden Semestern meines Studiums bei der YellowMap AG gelernt und getan habe. Dabei wird sich auf die Gesamtheit der Bemühungen begrenzt, die dazu galten das monolithische Serverdesign zu brechen und das bestehende System in ein verteiltes System zu wandeln. Zu diesem Zweck musste ich mich mit verschiedenen Arten der Serverkommunikation und Containerisierung beschäftigen. Im besonderen sollten dazu die beiden Technologien \gls{RabbitMQ} und \gls{Docker} verwendet werden. 

Auch sollten die Container durch \gls{Kubernetes} erzeugbar sein, so dass eine spätere dynamische Skalierung möglich wäre.

\paragraph{Vorab}
Das System ist als Monolith entworfen, der dem Model 2 Entwurfsmuster folgt, besteht aber aus zwei verschiedenen und getrennten Systemen, dem YM3 und YM4. 


\section{Serverkommunikation mit Message Broker} \label{rabbit}
Die erste Aufgabe, die ich bekam, war es die interne Serverkommunikation sicherstellen zu können. Damit sollten die zur Zeit getrennt laufenden Server nicht mehr darauf zurückgreifen sich über das externe Netz via \gls{HTTP}-Anfragen zu verständigen. Im Hinblick auf die späteren Änderungen, die am System vorgeführt werden mussten um es zu verteilen, entschied man sich für einen Message Broker.

\subsection{Was macht ein Message Broker?}
Ein Message Broker ist ein System zur Nachrichtenübertragung und Nachrichtenverteilung.

\subsection{Warum ein Message Broker? Warum RabbitMQ?}
In Hinblick auf die Containerisierung ist das Prinzip eines Message Brokers mehr als nützlich.
Ein Broker kann einen Nachrichten-Pool schaffen, den er durch routing keys gezielt mit Nachrichten versorgt, die dann von den letztendlichen Verbrauchern Konsumiert werden können.

Die Entscheidung dieses Prinzip mit RabbitMQ umzusetzen war eine einfache. RabbitMQ unterstützt nicht nur die komplette Spezifikation des \gls{AMQP}, es fügt auch nützliche Funktionen hinzu\cite{rabEx}.

\section{Containerisierung mit \gls{Docker}} \label{docker}
Warum Container? Warum Docker?

\paragraph{Vorgehen und Funde?}
docker docker docker

\section{Skallierung mit \gls{Kubernetes}} \label{kubernetes}
Warum überhaupt? Warum Kubernetes?

\paragraph{Vorgehen und Funde?}
docker docker docker docker docker docker

\section{Stand des Projekts} \label{project_status}
Erkläre halt wie weit du bist.

\section{Fazit}\label{conclusion}
Jetzt kannst du schwadronieren.
\newpage

%-----------------------------------------------------------------
% Ab hier => Literaturverzeichnis + temporäre Kommentare und Tests
%-----------------------------------------------------------------

\printbibliography
\newpage
\printglossary
\end{document}