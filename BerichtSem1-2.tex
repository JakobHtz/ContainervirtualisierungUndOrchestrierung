\documentclass[12pt,a4paper]{article}
\usepackage[utf8]{inputenc}% ermöglich die direkte Eingabe der Umlaute 
\usepackage[T1]{fontenc} % das Trennen der Umlaute
\usepackage[ngerman]{babel} % hiermit werden deutsche Bezeichnungen genutzt 
\usepackage{csquotes} 	% Gänsefüßchen nach gesetzter Sprache
\usepackage{graphicx}	% Bilder
\usepackage{subcaption}	% Für die Subfigures
\usepackage{geometry}	% Seitengröße
\usepackage[backend=biber,style=numeric,]{biblatex}	% Fürs Literatursverzeichis
\usepackage{glossaries}\makeglossaries
\geometry{
  left=3cm,
  right=3cm,
  top=2cm,
  bottom=4cm,
  bindingoffset=5mm
}
\loadglsentries{glossary.bib}
\addbibresource{literatur.bib}

%------------------------
% Deckblatt Titel und co.
%------------------------

\title{Praxisbericht\thanks{Damke am m1 Mamer uwu}}
\author{R\"uckblick auf die Semester 1 und 2\\
	Studiengang Informatik\\
	Studienrichtung Informatik\\
	DHBW Karlsruhe\\
	von Jakob Jonathan Heitzmann
}
\date{\today}

\begin{document}

%-------
% Bilder
%-------

\begin{figure}[t!]
	\begin{subfigure}{0.4\linewidth}
		\includegraphics[scale=0.25]{DHBWKarlsruhe.jpg}
	\end{subfigure}
	\hspace*{\fill}
	\begin{subfigure}{0.4\linewidth}
    	\includegraphics[scale=0.25]{YellowMap.png}
	\end{subfigure} 
\end{figure}

\maketitle
\vspace*{\fill}

%----------------
% Deckblatt unten
%----------------

\begin{center}
	\begin{tabular}{p{8cm}r}
		Matrikelnummer & 9119328\\
		Kurs & TINF19B4\\
		Ausbildungsfirma & YellowMap AG, Karlsruhe\\
		Betreuer & Markus Lind
	\end{tabular}
\end{center}
\newpage

%--------------------------
% Eidesstattliche Erklärung
%--------------------------

\section*{Eidesstattliche Erkl\"arung}

Gem\"a\ss\ \S\ 5 (3) der ''Studien- und Pr\"ufungsordnung DHBW Technik'' vom 29. September 2017 erkl\"are ich, Jakob Jonathan Heitzmann, dass ich die vorliegende Arbeit selbstst\"andig verfasst und keine anderen als die angegebenen Quellen und Hilfsmittel verwendet habe, sowie dass die physisch-vorliegende Version der Digitalen identisch ist.\\
\hspace*{0.5cm}
Karlsruhe, den \today

\vspace*{2cm}

\begin{tabular}{@{}l@{}}\hline
\rule{0pt}{2ex}
Jakob Jonathan Heitzmann
\end{tabular}
\newpage

%---------------
%Zusammenfassung
%---------------

\begin{abstract}
In dieser Arbeit geht es um meine Tätigkeit bei der YellowMap AG und was ich in den ersten zwei Semestern gelernt und getan habe. Diese Arbeit beschränkt sich aber in erster Linie auf die Bemühungen, die zur Umwandlung des Serversystems auf ein verteiltes System galten.
\end{abstract}
\newpage

\tableofcontents
\newpage

%-----------------------------------------------------------------------------------------------
%Hier beginnt das eigentliche Inhalt des Dokuments. (Davor nur Deckblatt bis Inhaltsverzeichniss)
%-----------------------------------------------------------------------------------------------

\section{Einf\"uhrung}
Ziel der Arbeit ist es vorzuzeigen was ich in den ersten beiden Semestern meines Studiums bei der YellowMap AG gelernt und getan habe. Dabei wird sich auf die Gesamtheit der Bemühungen begrenzt, die dazu galten die bestehende Serverstruktur zu brechen und einen Prototypen zu konzeptionieren, mit wessen Vorbild man das bestehende System in ein verteiltes System wandeln könnte. Zu diesem Zweck musste ich mich mit verschiedenen Arten der Serverkommunikation und Containerisierung beschäftigen. Im besonderen sollten dazu die beiden Technologien \gls{RabbitMQ} und \gls{Docker} verwendet werden. 

Auch sollten die Container durch \gls{Kubernetes} erzeugbar sein, so dass diese horizontal skaliert werden können.
Diese drei Technologien galten als gegeben und mussten verwendet werden. 

\paragraph{Das bisherige Serverdesign}
Das bestehende System wurde entworfen damit viele Anwendungen unabhängig voneinander entwickelt werden können ohne dass jede Funktionalität in jeder neuen Anwendung auch neu geschrieben werden müsste. Dazu greifen alle Projekte auf die selbe Schnittstelle zu, diese bietet dann die einzelnen Funktionalitäten in einem einheitlichen Format. Diese Art der Kommunikation ermöglicht es unabhängig von Art der Anwendung die Anbindung an die internen Dienste sicherzustellen. Eine Anfrage eines Projekt auf diese Schnittstelle landet 

Das bestehende System hat sich über die Jahre in zwei gespalten und besteht mittlerweile aus zwei verschiedenen Servern, dem Dreier-(YM3) und Vierersystem(YM4). Das Vierersystem ist das neuere System und war der Versuch das Dreiersystem zu ersetzen. Doch alle Funktionalitäten des Dreiersystems deckt das Vierersystem bis heute nicht ab und greift bei manchen Anfragen sogar selbst auf das Dreiersystem zurück. Dies läuft über eine gewöhnliche HTTP-Anfrage auf die API des Dreiersystems ab und läuft somit über das externe Netz. Wird also eine Anfrage auf das Vierersystem geschickt, dass es einen Ort mit Straße geocodieren solle, geht diese Nachricht als erstes an den Proxy des Vierersystems, der diese dann annimmt. Von hier aus geht sie zum System Call. Hier werden Partnerdaten ausgewertet und zum Beispiel auch der Provider für weitere Anfragen gesetzt. Nun wird ein Command oder eine Command Group aufgerufen, die dann die Daten weiterhin anreichern und dann eine Command Action aufruft. Diese führt dann die eigentliche Anfrage aus, aber in unserem Beispiel beinhaltet dies den Aufruf des Dreiersystems. Es wird eine Anfrage an das Dreiersystem gestellt, die dort angenommen und bis zur nächsten Command Action innerhalb des Dreiersystems geschleust wird, die dann endlich den Ort mit Straße geocodieren kann, womit eben Koordinaten und andere Informationen zu dem Gesuchten zurückgeliefert wird.

\section{Serverkommunikation mit Message Broker} \label{rabbit}
Die erste Aufgabe, die ich bekam, war es die interne Serverkommunikation sicherstellen zu können. Damit sollten die zur Zeit getrennt laufenden Server nicht mehr darauf zurückgreifen sich über das externe Netz via \gls{HTTP}-Anfragen zu verständigen. Im Hinblick auf die späteren Änderungen, die am System vorgenommen werden mussten um es zu verteilen, entschied man sich für einen Message Broker.

\subsection{Was macht ein Message Broker?}
Ein Message Broker ist ein System zur Nachrichtenübertragung und Nachrichtenverteilung. Dafür fungiert der Message Broker selbst als der zentrale Umschlagplatz von dem aus Nachrichten Empfangen und Verteilt werden. Ebenso übersetzt er aktiv zwischen verschiedenen Protokollen\cite{mesBro}. 

\subsection{Warum ein Message Broker? Warum RabbitMQ?}
\paragraph{Warum ein Message Broker}
In Hinblick auf die Containerisierung ist das Prinzip eines Message Brokers mehr als nützlich.
Ein Message Broker kann einen Nachrichten-Pool schaffen, den er kontinuierlich und gezielt mit Nachrichten versorgt, die dann von den letztendlichen Verbrauchern konsumiert werden können. Dadurch schafft man ein Netz von Geräten, die nur noch eine zentrale Adresse kennen müssen und dadurch dynamischer und unabhängiger voneinander agieren können. 

\paragraph{Warum RabbitMQ}
Die Entscheidung dieses Prinzip mit \gls{RabbitMQ} umzusetzen war eine Einfache. \gls{RabbitMQ} unterstützt nicht nur die komplette Spezifikation des \gls{AMQP}, es fügt auch nützliche Funktionen hinzu\cite{rabExt}.

\subsection{Implementierung}

\section{Containerisierung mit \gls{Docker}} \label{docker}
Docker ist ein Werkzeug zur Erstellung und Verwaltung von Containern. Container sind kleine voneinander unabhängige Umgebungen in denen Programme lauffähig und von außerhalb gezielt erreichbar sind.

\subsection{Was Containerisierung ist?}

\subsection{Warum Docker?}

\subsection{Umsetzung}

\section{Skalierung mit \gls{Kubernetes}} \label{kubernetes}
Kubernetes ist ein Werkzeug zur Erstellung und Verwaltung von Containerclustern. Containercluster sind Verbünde von einzelnen Containern die 

\subsection{Was ist Skalierung?}

\subsection{Warum \gls{Kubernetes}?} 

\subsection{Umsetzung}

\section{Stand des Projekts} \label{project_status}

\section{Fazit}\label{conclusion}

\newpage

%-----------------------------------------------------------------
% Ab hier => Literaturverzeichnis + temporäre Kommentare und Tests
%-----------------------------------------------------------------

\printbibliography
\newpage
\printglossary
\end{document}